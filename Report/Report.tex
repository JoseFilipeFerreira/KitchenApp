\documentclass[a4paper]{report}
\usepackage[utf8]{inputenc}
\usepackage[portuguese]{babel}
\usepackage{hyperref}
\usepackage{a4wide}
\hypersetup{pdftitle={LI4 - Proposta de Projeto},
pdfauthor={João Teixeira, José Ferreira, Miguel Solino, Maria Silva, Pedro
Oliveira},
colorlinks=true,
urlcolor=blue,
linkcolor=black}
\usepackage{subcaption}
\usepackage[cache=false]{minted}
\usepackage{listings}
\usepackage{booktabs}
\usepackage{multirow}
\usepackage{appendix}
\usepackage{tikz}
\usepackage{authblk}
\usepackage{bashful}
\usepackage{verbatim}
\usepackage{amsmath}
\usepackage{tikz}
\usepackage{tikz,fullpage}
\usepackage{pgfgantt}
\usetikzlibrary{arrows,%
                petri,%
                topaths}%
\usepackage{tkz-berge}
\usetikzlibrary{positioning,automata,decorations.markings}
\AfterEndEnvironment{figure}{\noindent\ignorespaces}
\AfterEndEnvironment{table}{\noindent\ignorespaces}

\begin{document}

\title{LI4 - Proposta de Projeto\\ 
\large PL1 - Grupo 1.1}
\author{José Ferreira (A83683) \and João Teixeira (A85504) \and Maria Silva
(A83840) \and Miguel Solino (A86435) \and Pedro Oliveira (A83762)}
\date{\today}

\begin{center}
    \begin{minipage}{0.75\linewidth}
        \centering
        \includegraphics[width=0.4\textwidth]{images/eng.jpeg}\par\vspace{1cm}
        \vspace{1.5cm}
        \href{https://www.uminho.pt/PT}
        {\color{black}{\scshape\LARGE Universidade do Minho}} \par
        \vspace{1cm}
        \href{https://www.di.uminho.pt/}
        {\color{black}{\scshape\Large Departamento de Informática}} \par
        \vspace{1.5cm}
        \maketitle
    \end{minipage}
\end{center}

\pagebreak
\section{Título e Enquadramento}
Um dos grandes problemas de gestão que todas as pessoas enfrentam ao longo da
sua vida é como gerir as suas compras de produtos alimentares. É necessário
gerir o que comprar, quando comprar e como comprar de forma a manter a dispensa
recheada. E os problemas não acabam por aí, depois de comprar os produtos é
necessário decidir o que fazer com eles e impedir que eles se estraguem. Com as
nossas vidas cada vez mais ocupadas e privadas de tempo, algo simples que junte
estas funcionalidades todas seria uma lufada de ar fresco.
The Kitchen App

\section{Objetivos}
O objetivo deste projeto é criar uma aplicação que facilite a gestão e
aproveiamente da despensa em casa permitindo também a criação de despensas
partilhadas que podem ser editadas por todos os residentes.
Para facilitar o aproveitamento dos ingredientes a aplicação irá sugerir várias
receitas que poderão posteriormente ser adicionadas aos favoritos. De forma a
diminuir o desperdício alimentar a aplicação também informará o utilizador
quando os produtos estão quase a ficar fora do prazo de validade.

\section{Requisitos}
\subsection{Requisitos Funcionais}
\begin{itemize}
    \item Inserir/Remover produtos na dispensa
    \item Criação de Utilizador
    \item Procurar receitas com base nos produtos desponíveis
    \item Procurar receitas
    \item Procurar produtos
    \item criar lista de compras
    \item Listar receitas favoritas
    \item Criar despensa
    \item Adicionar utilizador a uma despensa
\end{itemize}

\subsection{Requisitos não-funcionais}
\begin{itemize}
    \item Os produtos devem ser adicionados com o nome, tipo e data de validade
        (opcional)
    \item Um utilizador deve ser criado fornecendo um nome, password, e-mail e
        data de nascimento
    \item Na procura de produtos pode ser fornecido o tipo a pesquisar
    \item A criação de uma lista de compras pode ser feito com base numa receita
    \item Receitas podem ser adicionadas por utilizadores de forma pública ou
        privada
    \item Um utilizador pode ter receitas favoritas
    \item As receitas pesquisadas podem ser ordenadas com base nos produtos
        disponíveis na dispensa
\end{itemize}

\section{Tecnologias}
\begin{itemize}
    \item C\#
    \item MySQL
    \item JavaScript
    \item HTML/CSS
    \item Python
    \item Bash
\end{itemize}

\section{Cronograma}
\begin{figure}[H]
    \centering
    \begin{ganttchart}[vgrid, hgrid]{1}{27}
        \ganttbar{Task 0}{1}{4}\\
        \ganttbar{Task 6}{1}{5}\\

        \ganttbar{Task 1}{5}{10}\\
        \ganttbar{Task 7}{6}{11}\\
        \ganttbar{Task 10}{6}{13}\\
        
        \ganttbar{Task 4}{12}{20}\\
        \ganttbar{Task 11}{14}{20}\\
        
        \ganttbar{Task 8}{17}{20}\\
        
        \ganttbar{Task 9}{21}{22}\\
        
        \ganttbar{Task 2}{21}{27}\\
    \end{ganttchart}
    \caption{diagrama de Gantt}
\end{figure}

\end{document}
