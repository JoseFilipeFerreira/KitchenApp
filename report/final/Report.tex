\documentclass[a4paper]{report}
\usepackage[utf8]{inputenc}
\usepackage[portuguese]{babel}
\usepackage{hyperref}
\usepackage{a4wide}
\hypersetup{pdftitle={LI4 - PL1 - Grupo 1},
pdfauthor={João Teixeira, José Ferreira, Maria Silva, Miguel Solino, Pedro Oliveira},
colorlinks=true,
urlcolor=blue,
linkcolor=black}
\usepackage{subcaption}
\usepackage[cache=false]{minted}
\usepackage{listings}
\usepackage{booktabs}
\usepackage{multirow}
\usepackage{appendix}
\usepackage{tikz}
\usepackage{authblk}
\usepackage{bashful}
\usepackage{verbatim}
\usepackage{amsmath}
\usepackage{amssymb}
\usepackage{multirow}
\usepackage{mwe}
\usetikzlibrary{arrows,%
                petri,%
                topaths}%
\usetikzlibrary{positioning,automata,decorations.markings}
\AfterEndEnvironment{figure}{\noindent\ignorespaces}
\AfterEndEnvironment{table}{\noindent\ignorespaces}

\definecolor{solarized@base03}{HTML}{002B36}
\definecolor{solarized@base02}{HTML}{073642}
\definecolor{solarized@base01}{HTML}{586e75}
\definecolor{solarized@base00}{HTML}{657b83}
\definecolor{solarized@base0}{HTML}{839496}
\definecolor{solarized@base1}{HTML}{93a1a1}
\definecolor{solarized@base2}{HTML}{EEE8D5}
\definecolor{solarized@base3}{HTML}{FDF6E3}
\definecolor{solarized@yellow}{HTML}{B58900}
\definecolor{solarized@orange}{HTML}{CB4B16}
\definecolor{solarized@red}{HTML}{DC322F}
\definecolor{solarized@magenta}{HTML}{D33682}
\definecolor{solarized@violet}{HTML}{6C71C4}
\definecolor{solarized@blue}{HTML}{268BD2}
\definecolor{solarized@cyan}{HTML}{2AA198}
\definecolor{solarized@green}{HTML}{859900}

\lstset{
  language=Java,
  upquote=true,
  columns=fixed,
  tabsize=4,
  extendedchars=true,
  breaklines=true,
  numbers=left,
  numbersep=5pt,
  rulesepcolor=\color{solarized@base03},
  numberstyle=\tiny\color{solarized@base01},
  basicstyle=\footnotesize\ttfamily,
  keywordstyle=\color{solarized@green},
  stringstyle=\color{solarized@yellow}\ttfamily,
  identifierstyle=\color{solarized@blue},
  commentstyle=\color{solarized@base01},
  emphstyle=\color{solarized@red}
}

\begin{document}

\title{Laboratórios de Informática IV\\
\large PL1 - Grupo 1}
\author{José Ferreira (A83683) \and João Teixeira (A85504) \and Maria Silva (A83840) \and Miguel Solino (A86435) \and Pedro Oliveira (A83762)}
\date{\today}

\begin{center}
    \begin{minipage}{0.75\linewidth}
        \centering
        \includegraphics[width=0.4\textwidth]{images/eng.jpeg}\par\vspace{1cm}
        \vspace{1.5cm}
        \href{https://www.uminho.pt/PT}
        {\color{black}{\scshape\LARGE Universidade do Minho}} \par
        \vspace{1cm}
        \href{https://www.di.uminho.pt/}
        {\color{black}{\scshape\Large Departamento de Informática}} \par
        \vspace{1.5cm}
        \maketitle
    \end{minipage}
\end{center}


\chapter{Resumo}

\tableofcontents

\chapter{Introdução}
    \section{Contextualização}
    \section{Apresentação do Caso de Estudo}
    \section{Motivação e Objetivos}
    \section{Estrutura do Relatório}

\chapter{Fundamentação}
    \section{Definição da Identidade do Sistema}
    \section{Justificação, Viabilidade e Utilidade do Sistema}

\chapter{Planeamento}
    \section{Identificação dos Recursos Necessários}
    \section{Modelação do Sistema a Implementar - Maqueta}
    \section{Definição de Medida de Sucesso}
    \section{Plano de Desenvolvimento}

\chapter{Levantamento de Requisitos}
    \section{Requisitos Funcionais}
        \subsection{Gestão de Utilizador}
            \textbf{Definição de Requisitos de Utilizador}
            \textbf{Definição de Requisitos de Sistema}
            
        \subsection{Gestão de Despensas}
            \textbf{Definição de Requisitos de Utilizador}
            \textbf{Definição de Requisitos de Sistema}

        \subsection{Gestão de Wishlists}
            \textbf{Definição de Requisitos de Utilizador}
            \textbf{Definição de Requisitos de Sistema}

        \subsection{Gestão de lista de compras}
            \textbf{Definição de Requisitos de Utilizador}
            \textbf{Definição de Requisitos de Sistema}

        \subsection{Adicionar utilizador a uma despensa}
            \textbf{Definição de Requisitos de Utilizador}
            \textbf{Definição de Requisitos de Sistema}

        \subsection{Gestão de produtos}
            \textbf{Definição de Requisitos de Utilizador}
            \textbf{Definição de Requisitos de Sistema}

        \subsection{Gestão de Receitas Favoritas}
            \textbf{Definição de Requisitos de Utilizador}
            \textbf{Definição de Requisitos de Sistema}

        \subsection{Procura de Produtos}
            \textbf{Definição de Requisitos de Utilizador}
            \textbf{Definição de Requisitos de Sistema}

        \subsection{Procura de Receitas}
            \textbf{Definição de Requisitos de Utilizador}
            \textbf{Definição de Requisitos de Sistema}

        \subsection{Gestão de amizades}
            \textbf{Definição de Requisitos de Utilizador}
            \textbf{Definição de Requisitos de Sistema}


    \section{Requisitos Não Funcionais}
    \begin{itemize}
        \item Um utilizador deve ser criado fornecendo um nome, password, email e data de nascimento
        \item Uma despensa, wishlist e lista de compras devem ser criadas com nome
        \item Os produtos devem ser adicionados com nome, quantidade e data de validade
        \item As receitas podem ser pesquisadas conforme os produtos de uma dispensa
        \item Um utilizador pode ter receitas favoritas
        \item Um utilizador pode ser amigo de um ou mais utilizadores
        \item A aplicação deverá estar disponível durante os 7 dias da semanas, 24 horas por dia
    \end{itemize}

\chapter{Modelo de Domínios}

\chapter{Diagrama de Use Cases}
    \begin{figure}[H]
    \centering
        \includegraphics[width=\textwidth]{images/diagrama_use_cases.png}
    \end{figure}

\chapter{Especificação de Use Cases}
    \section{Registar}
        \begin{figure}[H]
        \centering
            \includegraphics[width=\textwidth]{images/usecases/registar.png}
        \end{figure}

    \section{Autenticar}
        \begin{figure}[H]
        \centering
            \includegraphics[width=\textwidth]{images/usecases/autenticar.png}
        \end{figure}

    \section{Editar informação pessoal}
        \begin{figure}[H]
        \centering
            \includegraphics[width=\textwidth]{images/usecases/editar_utilizador.png}
        \end{figure}

    \section{Editar receitas favoritas}
        \begin{figure}[H]
        \centering
            \includegraphics[width=\textwidth]{images/usecases/editar_receitas_favoritas.png}
        \end{figure}

    \section{Ver perfil}
        \begin{figure}[H]
        \centering
            \includegraphics[width=\textwidth]{images/usecases/ver_perfil.png}
        \end{figure}

    \section{Editar inventários}
        \begin{figure}[H]
        \centering
            \includegraphics[width=\textwidth]{images/usecases/editar_iventarios_1.png}
            \includegraphics[width=\textwidth]{images/usecases/editar_iventarios_2.png}
        \end{figure}

    \section{Editar wishlists}
        \begin{figure}[H]
        \centering
            \includegraphics[width=\textwidth]{images/usecases/editar_whishlist_1.png}
            \includegraphics[width=\textwidth]{images/usecases/editar_whishlist_2.png}
        \end{figure}

    \section{Editar listas de compras}
        \begin{figure}[H]
        \centering
            \includegraphics[width=\textwidth]{images/usecases/editar_shoppinglist_1.png}
            \includegraphics[width=\textwidth]{images/usecases/editar_shoppinglist_2.png}
        \end{figure}

    \section{Editar amigos}
        \begin{figure}[H]
        \centering
            \includegraphics[width=\textwidth]{images/usecases/editar_amigos.png}
        \end{figure}

    \section{Pesquisar produtos}
        \begin{figure}[H]
        \centering
            \includegraphics[width=\textwidth]{images/usecases/pesquisar_produto.png}
        \end{figure}

    \section{Ver receita}
        \begin{figure}[H]
        \centering
            \includegraphics[width=\textwidth]{images/usecases/ver_receita.png}
        \end{figure}

\chapter{Máquina de Estado - Preparar Receita}

\chapter{Arquitetura da Camada de Negócios}
    \section{Camada de Negócios}
    \section{Dicionário das Principais Classes}
        \subsection{User}
        Corresponde à representaão no sistema dos utilizadoeres registados,
        contendo a sua informação pessoal e credenciais de acesso.
        \subsection{Product}
        Representa a unidade básica de todas as dispensas, listas de compras e
        listas de favoritos. Contém informação do nome, categoria, preço e
        quantidades por embalagem. Existem classes referentes aos vários tipos
        de produtos, que herdam desta, e contêm informação extra para ser
        possível acomodar os diferentes tipos de listas presentes no programa.
        \subsection{Inventory}
        A classe Inventory representa uma lista de produtos de um utilizador, contendo
        informação sobre o seu dono, nome e produtos disponíveis nesta, e 
        com quem é partilhada. A lista de produtos tem que pertencer à super
        classe \textit{Product}, e vai ser isto que vai determinar a que
        corresponde o inventário em questão, se é uma dispensa, lista de
        favoritos ou de compras.
        \subsection{Recipe}
        Esta classe representa uma receita, contendo informação sobre o seu
        nome, resumo, ingredientes necessários à confeção e instruções de
        preparação.
    \section{Descrição da Arquitetura}
    Cada utilizador possui uma lista de dispensas, uma lista de listas de 
    produtos favoritos, uma lista de listas de compras e uma lista de receitas
    favoritas.
    Também existe a possibilidade de um utilizador partilhar as suas listas de
    produtos, tendo cada utilizador uma lista de inventários partilhados
    consigo.
    \section{Diagrama de Classes}
        \begin{figure}[H]
        \centering
                \includegraphics[width=\textwidth]{images/diagrama_de_classes.png}
        \end{figure}

    \section{Diagrama de ORM}
        \begin{figure}[H]
        \centering
            \includegraphics[width=\textwidth]{images/diagrama_de_ORN.png}
        \end{figure}

\chapter{Camada de Dados}
    \section{Modelo Lógico}
    \section{Diagrama de Modelo Lógico}

\chapter{Proposta de Interface}
    \begin{figure}[H]
    \centering
            \includegraphics[width=0.7\textwidth]{images/mockup/login_desktop.png}
            \caption{Login no Desktop}
    \end{figure}
    \begin{figure}[H]
    \centering
            \includegraphics[width=0.2\textwidth]{images/mockup/login_mobile.png}
            \caption{Login em Mobile}
    \end{figure}
    \begin{figure}[H]
    \centering
            \includegraphics[width=0.7\textwidth]{images/mockup/register_desktop.png}
            \caption{Registo no Desktop}
    \end{figure}
    \begin{figure}[H]
    \centering
            \includegraphics[width=0.2\textwidth]{images/mockup/register_mobile.png}
            \caption{Registo em Mobile}
    \end{figure}
    \begin{figure}[H]
    \centering
            \includegraphics[width=0.7\textwidth]{images/mockup/dashboard_desktop.png}
            \caption{Dashboard no Desktop}
    \end{figure}
    \begin{figure}[H]
    \centering
            \includegraphics[width=0.2\textwidth]{images/mockup/dashboard_mobile.png}
            \caption{Dashboard em Mobile}
    \end{figure}
    

\chapter{Metodologia de Implementação}
O padrão arquitetural escolhido para a implementação da aplicação foi o MVC
(\textit{Model View Controller}), por forma a obter um conjunto de pequenos
componentes modulares de fácil integração, e assim facilitar o desenvolvimento 
em paralelo. De forma a tornar a implementação ainda mais modular, a componente
do controlador foi decomposta em pequenos controladores individuais, cada um
responsável por lidar com uma pequena parte da aplicação, como por exemplo,
lidar com as dispensas, ou com a autenticação.

A escolha deste padrão arquitetural proporciona também uma camada de abstação,
deixando a possibilidade de implementar a aplicação em mais do que uma
plataforma, sem fazer alterações ao préviamente feito.

A componente do modelo foi decomposta em camada de negócio e de dados, por forma
a garantir a consitência, integridade e disponibilidade dos dados da aplicação,
qualquer que seja a tecnologia utilizada para o suporte da base de dados.

\chapter{Ferramentas utilizadas na implementação}

A principal ferramenta utilizada para a implementação da aplicação apresentada
foi a \textit{framework} de desenvolvimento ASP.NET Core. Esta escolha foi
tomada graças à excelente documentação disponível e simplicidade de utilização.

Como gestor de base de dados escolhemos o \textit{Neo4j}, que utiliza um modelo
não relacional e que se enquadra com a disposição definida dos dados da
aplicação.

Para a criação da interface do produto final, foi utilizada a biblioteca
\textit{React}, assente na linguagem de programação \textit{JavaScript}, pela
sua popularidade e facilidade de utilização.

Para tornar o \textit{deployment} da aplicação o mais agnóstico da máquina onde
é feito, e para facilitar os testes feitos durante o processo de
desenvolvimento, foi utilizada a ferramenta \textit{Docker} e \textit{Docker
Compose}, na qual foram criados \textit{containers} contendo a aplicação, todas
as suas configurações, tornando o processo de correr a aplicação apenas um
comando, sem haver qualquer preocupação ao nível de \textit{networking}.

Para garantir a disponibilidade da aplicação a qualquer hora, esta foi hospedada
na plataforma \textit{Azure}, tirando partido do esforço anterior de utilização
da ferramenta \textit{Docker}.

Por fim, foram utilizados diversos \textit{browsers}, como o \textit{Firefox} ou
o \textit{Brave}, em várias plataformas, para garantir a boa aparência da
interface apresentada ao utilizador final.

\chapter{Desenvolvimento do Projeto}
    \section{Conexão da Base de Dados}
    \section{API de Receitas}

\chapter{Produto Final}
    \begin{figure}[H]
        \centering
            \includegraphics[width=\textwidth]{images/produto_final/login.png}
    \end{figure}

    \begin{figure}[H]
        \centering
            \includegraphics[width=\textwidth]{images/produto_final/registo.png}
    \end{figure}

    \begin{figure}[H]
        \centering
            \includegraphics[width=\textwidth]{images/produto_final/inicial.png}
    \end{figure}

    \begin{figure}[H]
        \centering
            \includegraphics[width=\textwidth]{images/produto_final/iventario.png}
    \end{figure}

    \begin{figure}[H]
        \centering
            \includegraphics[width=\textwidth]{images/produto_final/iventario_receitas.png}
    \end{figure}

    \begin{figure}[H]
        \centering
            \includegraphics[width=\textwidth]{images/produto_final/inserir_produto_categoria.png}
    \end{figure}

    \begin{figure}[H]
        \centering
            \includegraphics[width=\textwidth]{images/produto_final/inserir_produto_produto.png}
    \end{figure}

    \begin{figure}[H]
        \centering
            \includegraphics[width=\textwidth]{images/produto_final/inserir_produto_quantidade.png}
    \end{figure}

    \begin{figure}[H]
        \centering
            \includegraphics[width=\textwidth]{images/produto_final/alterar_nome_iventario.png}
    \end{figure}

    \begin{figure}[H]
        \centering
            \includegraphics[width=\textwidth]{images/produto_final/partilhar_com_outro_utilizador.png}
    \end{figure}

    \begin{figure}[H]
        \centering
            \includegraphics[width=\textwidth]{images/produto_final/editar_produto.png}
    \end{figure}

    \begin{figure}[H]
        \centering
            \includegraphics[width=\textwidth]{images/produto_final/eliminar_produto.png}
    \end{figure}

    \begin{figure}[H]
        \centering
            \includegraphics[width=\textwidth]{images/produto_final/pagina_iventario_partilhado.png}
    \end{figure}

    \begin{figure}[H]
        \centering
            \includegraphics[width=\textwidth]{images/produto_final/procura_de_produtos.png}
    \end{figure}

    \begin{figure}[H]
        \centering
            \includegraphics[width=\textwidth]{images/produto_final/procura_de_produtos_efetuada.png}
    \end{figure}

    \begin{figure}[H]
        \centering
            \includegraphics[width=\textwidth]{images/produto_final/procura_de_receitas.png}
    \end{figure}

    \begin{figure}[H]
        \centering
            \includegraphics[width=\textwidth]{images/produto_final/procura_de_receitas_efetuadas.png}
    \end{figure}

    \begin{figure}[H]
        \centering
            \includegraphics[width=\textwidth]{images/produto_final/receitas_favoritas.png}
    \end{figure}

    \begin{figure}[H]
        \centering
            \includegraphics[width=\textwidth]{images/produto_final/pagina_de_perfil.png}
    \end{figure}

    \begin{figure}[H]
        \centering
            \includegraphics[width=\textwidth]{images/produto_final/alterar_nome_perfil.png}
    \end{figure}

    \begin{figure}[H]
        \centering
            \includegraphics[width=\textwidth]{images/produto_final/alterar_numero_perfil.png}
    \end{figure}

    \begin{figure}[H]
        \centering
            \includegraphics[width=\textwidth]{images/produto_final/alterar_nascimento_iventario.png}
    \end{figure}

    \begin{figure}[H]
        \centering
            \includegraphics[width=\textwidth]{images/produto_final/pedido_amigos_enviado.png}
    \end{figure}

    \begin{figure}[H]
        \centering
            \includegraphics[width=\textwidth]{images/produto_final/adicionar_amigo.png}
    \end{figure}

    \begin{figure}[H]
        \centering
            \includegraphics[width=\textwidth]{images/produto_final/pedido_recebido.png}
    \end{figure}

    \begin{figure}[H]
        \centering
            \includegraphics[width=\textwidth]{images/produto_final/amigo_adicionado.png}
    \end{figure}

    \begin{figure}[H]
        \centering
            \includegraphics[width=\textwidth]{images/produto_final/exemplo_erro.png}
    \end{figure}

    \begin{figure}[H]
        \centering
            \includegraphics[width=\textwidth]{images/produto_final/exemplo_sucesso.png}
    \end{figure}

    \begin{figure}[H]
        \centering
            \includegraphics[width=\textwidth]{images/produto_final/notificacoes.png}
    \end{figure}

\chapter{Conclusões}

\chapter{Referências}

\chapter{Lista de Siglas e Acrónimos}

\end{document}
