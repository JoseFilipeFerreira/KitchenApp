\documentclass[a4paper]{report}
\usepackage[utf8]{inputenc}
\usepackage[portuguese]{babel}
\usepackage{hyperref}
\usepackage{a4wide}
\hypersetup{pdftitle={Trabalho 1},
pdfauthor={João Teixeira, José Ferreira, Maria Silva, Miguel Solino, Pedro Oliveira},
colorlinks=true,
urlcolor=blue,
linkcolor=black}
\usepackage{subcaption}
\usepackage[cache=false]{minted}
\usepackage{listings}
\usepackage{booktabs}
\usepackage{multirow}
\usepackage{appendix}
\usepackage{tikz}
\usepackage{authblk}
\usepackage{bashful}
\usepackage{verbatim}
\usepackage{amsmath}
\usetikzlibrary{positioning,automata,decorations.markings}
\definecolor{solarized@base03}{HTML}{002B36}
\definecolor{solarized@base02}{HTML}{073642}
\definecolor{solarized@base01}{HTML}{586e75}
\definecolor{solarized@base00}{HTML}{657b83}
\definecolor{solarized@base0}{HTML}{839496}
\definecolor{solarized@base1}{HTML}{93a1a1}
\definecolor{solarized@base2}{HTML}{EEE8D5}
\definecolor{solarized@base3}{HTML}{FDF6E3}
\definecolor{solarized@yellow}{HTML}{B58900}
\definecolor{solarized@orange}{HTML}{CB4B16}
\definecolor{solarized@red}{HTML}{DC322F}
\definecolor{solarized@magenta}{HTML}{D33682}
\definecolor{solarized@violet}{HTML}{6C71C4}
\definecolor{solarized@blue}{HTML}{268BD2}
\definecolor{solarized@cyan}{HTML}{2AA198}
\definecolor{solarized@green}{HTML}{859900}

\lstset{
  language=Python,
  upquote=true,
  columns=fixed,
  tabsize=4,
  extendedchars=true,
  breaklines=true,
  numbers=left,
  numbersep=5pt,
  rulesepcolor=\color{solarized@base03},
  numberstyle=\tiny\color{solarized@base01},
  basicstyle=\footnotesize\ttfamily,
  keywordstyle=\color{solarized@green},
  stringstyle=\color{solarized@yellow}\ttfamily,
  identifierstyle=\color{solarized@blue},
  commentstyle=\color{solarized@base01},
  emphstyle=\color{solarized@red}
}

%%%%    TECNOLOGIAS UTILIZADAS    %%%%
%   Software:
%   [?] Android Studio - App para criar a aplicação android
%   [?] Postman - Testar pedidos, Verificar headers
%   Visual Studio
%   
%   Linguagens Programação:
%   C#
%   Javascript
%   CSS
%   [?] Python
%
%   Base Dados:
%   Neo4J
%   
%   Outros:
%   HTML
%   React - Library de javascript para construir interfaces

\begin{document}

% Title
\title{Titulo\\
\large Grupo Nº 1}

% Author
\author{João Teixeira (A85504) \and José Ferreira (A83683) \and Maria Silva (A83840) \and Miguel Solino (A86435) \and Pedro Oliveira (A83762)}

% % Supervisor(s) (!!!)
% \supervisor{Orientador}
% \cosupervisor{Coorientador}

% Date
\date{\today}

% (!!!) linhas 98 - 132 ?
\begin{center}
    \begin{minipage}{0.75\linewidth}
        \centering
       
% (!!!) mudar imagem       
        \includegraphics[width=0.6\textwidth]{uminho.png}\par\vspace{1cm}
        \vspace{1cm}
        \href{https://www.uminho.pt/PT}
        {\color{black}{\scshape\LARGE Universidade do Minho}} \par
        \vspace{1cm}
        \href{https://www.di.uminho.pt/}
        {\color{black}{\scshape\Large Departamento de Informática}} \par
        \maketitle
    \end{minipage}
\end{center}

\tableofcontents

\pagebreak

% CHAPTER - Introdução -------------------------------------------------------------------------------
	\chapter{Introduction}
	    \input{introduction/intro.tex}
        \section{Context}
        % https://www.creditdonkey.com/grocery-shopping-statistics.html#:~:text=According%20to%20the%20Food%20Marketing,of%201.5%20times%20per%20week.
% https://www.statista.com/statistics/245501/multiple-mobile-device-ownership-worldwide/
\paragraph{}
De acordo com o 'Food market Institute' uma pessoa em média visita o Super Mercado 1.5 vezes por semana, ou seja, por volta de 72 vezes por ano. Devido a este facto diversas aplicações e sites de listas de compras foram criadas para facilitar a realização das compras, várias com milhões de transferências.
\par
De momento não existe uma aplicação ou site que permita a gestão de dispensa, em que, ao mesmo tempo, tenha a funcionalidade de uma lista de compras. Atualmente a utilização do \textit{smartphone} é considerado banal. É estimado que existam 1.8 biliões de telémoveis, o que, para ser possível distribuir um produto ao maior numero de consumidores possível, a melhor forma seria a utilização de um site que posteriormente poderia ser transformado numa aplicação.
        \section{Motivation}
        \input{introduction/motiv.tex}
        \section{Goals}
        \paragraph{}
The Kitchen App tem como objetivo fornecer ao utilizador uma forma fácil de organizar uma ou mais dispensas, procurar receitas, bem como verificar a presença de um produto numa determinada dispensa. Também é possivel adicionar produtos a uma lista de compras quer seja produtos individuais, ou produtos presentes numa receita.
\par

%%NOTA, ISTO É A BASE AINDA FALTA IMENSA COISA
        \section{Structure}
        \input{introduction/structure.tex}


% CHAPTER - Estado da Arte ---------------------------------------------------------------------------
	\chapter{Literature Review}
		State of the art review; related work
        \input{literaturereview/introduction.tex}
        \input{literaturereview/section1.tex}
        \section{Section 1}

        
        \section{Basics/Background/Related work}

	% CHAPTER - Problem and Challenges ---------------
	\chapter{Methods}
	         The problem and its challenges.

	\section{Proposed Approach - solution}
	In this section, it is presented various ways to display an image.
     \subsection{System Architecture}
     A block diagram of the planned system / approach

	Here we have an example of inserting an image between the text paragraphs.
	
	% CHAPTER - Contribution -------------------------
	\chapter{Development}
		
	\section{Decisions}
    \section{Implementation}
    \section{Outcomes}
    Main result(s) and their scientific evidence
	\section{Summary}

	% CHAPTER - Application -------------------------
	\chapter{Case Studies / Experiments}
		Application of main result (examples and case studies)
	\section{Experiment setup}
    \section{Results}
    \section{Discussion}
	\section{Summary}

	% CHAPTER - Conclusion/Future Work --------------
	\chapter{Conclusion}
		Conclusions and future work.
	\section{Conclusions}
	\section{Prospect for future work}
	
	% (!!!)
	\bookmarksetup{startatroot} % Ends last part.
	\addtocontents{toc}{\bigskip} % Making the table of contents look good.
	%\cleardoublepage

	%- Bibliography (needs bibtex) -%
	\bibliography{dissertation}

	% Index of terms (needs  makeindex) -------------
	%\printindex
	
	
	% APPENDIX --------------------------------------
	\umappendix{Appendix}
	
	% Add appendix chapters
	\chapter{Support material}
	Auxiliary results which are not main-stream; or

	%\chapter{Details of results}
	Details of results whose length would compromise readability of main text; or

	%\chapter{Listings}
	Specifications and Code Listings: should this be the case; or

	%\chapter{Tooling}
	Tooling: Should this be the case.

	%Anyone using \Latex\ should consider having a look at \TUG,
	%the \tug{\TeX\ Users Group}


	% Back Cover -------------------------------------------
	% (!!!)
	\umbackcover{
	NB: place here information about funding, FCT project, etc in which the work is framed. Leave empty otherwise.
	}

\end{document}